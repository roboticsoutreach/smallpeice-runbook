\section{Talks and Lectures}

\subsection{Introductory talk}

The introductory talk is the first activity of the week and is broken down into several sections from different speakers.

\begin{table}[h]
\begin{tabular}{ll}
 Talk & Speaker \\
 \hline
 Housekeeping & ECS course coordinator \\
 Smallpeice Welcome &  Smallpeice course coordinator \\
 ECS Welcome & Head of ECS or similar \\
 How to build your robot & Sourcebots representative \\
 Game rules & Head Judge \\
\end{tabular}
\caption{Speakers for each section of the introductory talk}
\end{table}

\subsubsection{Housekeeping}

The housekeeping talk covers important information for the course participants lasting around 5 minutes.
The main areas to cover are health and safety and the structure of the following talks.

\subsubsection{Smallpeice Welcome}

This is the welcome from Smallpeice and covers the basic structure of the course along with the rules for participants.
This talk lasts around 15 minutes and is given by the Smallpeice course coordinator.

\subsubsection{ECS Welcome}

This talk provides some information on the university and the courses available.
A representative from ECS (Often the head of school, or Director of outreach) will give this talk,
ideally the speaker will also be present at the formal meal and the prize giving at the end of the course.
This section should last around 15 minutes.


\subsubsection{How to build your robot}

This talk goes through the basics of robot assembly and the main activities during the week.
This talk will last around half an hour.

The talk starts by introducing the Sourcebots volunteers, each saying a bit about themselves and what they do.
After introductions the basic activities of workshops, lectures and robot hacking are introduced.
The talk then goes through common pitfalls and problems providing information on good robot design.
The kit and the component parts are then introduced along with safety information about batteries and tool use.

\subsubsection{Game Rules}

The final talk of the section is the game rules briefing by the head judge.
Half an hour has been allocated for the talk, however, if it ends early the extra time can be used for the teams to start planning their robots.

This talk should go through how the game works and the game rules,
a Q\&A session at the end provides a chance for teams to confirm any queries before they start designing their robot.



\task{Confirm presenter for ECS welcome presentation}
\task{Prepare robot building presentation}
\task{Prepare rules presentation}
\task{Print rules documents}
\task{Pack bags of pens etc. for students}
\task{Have robotics kit ready for demoing}

\subsection{Lectures}

There are three lectures during the course on topics that are relevant to the students such as computer vision.
Each of the lectures will be about an hour long and are given by a lecturer at the university.

\task{Select lecturers for lecture slots}
\task{Book rooms for lectures}

\subsection{Admissions talk}

During the formal meal an admissions advice talk is given by the ECS admissions tutors or suitable alternative.
This talk provides information on the admissions process and courses at Southampton and other universities.

\task{Confirm admissions tutors for talk}
